% -*-coding: utf-8 -*-

\zp{Básnířka}{Jarek Nohavica}
\zs
\Ch{G}{Mladičká} \Ch{D}{bás}nířka \Ch{C}{s} korálky \Ch{D}{nad} kotní\Ch{G}{ky} \Ch{D}{} \Ch{C}{} \Ch{D}{}
\Ch{G}{bouchala} \Ch{D}{na} dvířka \Ch{C}{paláce} \Ch{D}{poe}tiky \Ch{Emi}{}  \Ch{G}{}

S někým se \Ch{C}{vyspa}\Ch{G}{la}, někomu \Ch{C}{ned}\Ch{G}{ala},
láska jako \Ch{D}{hobby}

\Ch{G}{Pak} o tom \Ch{D}{nap}sala \Ch{C}{sonetu} na \Ch{D}{čtyř}i \Ch{G}{doby}  \Ch{D}{} \Ch{C}{} \Ch{D}{}
\ks
\zs
Svoje srdce skloňovala podle vzoru Ferlinghetti,
ve vzduchu nechávala viset vždy jen půlku věty

Plná tragiky, plná mystiky,
plná splínu

Tak jí to otiskli v jednom magazínu
\ks
\zs
Bývala viděna v malém baru u Rozhlasu,
od sebe kolena a cizí ruka kolem pasu

Trochu se napila, trochu se opila,
na účet redaktora

Za týden na to byla hvězdou mikrofóra
\ks
\zs
Pod paží nosila rozepsané rukopisy,
ráno se budila vedle záchodové mísy

Múzou políbena, životem potřísněná,
plná zázraků

A pak ji vyhodili z gymplu i z baráku
\ks
\zs
Šly řeči okolím, že měla něco se esenbáky,
ať bylo cokoliv, přestala věřit na zázraky

Cítila u srdce, jak po ní přešla,
železná bota

Tak o tom napsala sonet ze života
\ks
\zs
Pak jednou v pondělí, přišla na koncert na koleje,
a hlasem pokorným prosila o text Darmoděje

Péro vzala, pak se dala,
tichounce do pláče

/: A její slzy kapaly na její mrkváče :/
\ks
\kp





