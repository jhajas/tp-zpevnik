% -*-coding: utf-8 -*-

\zp{Zítra ráno v pět}{Jarek Nohavica}

\zs
Až mě \Ch{Emi}{zítra} ráno v pět
\Ch{G}{ke zdi} postaví,

je\Ch{C}{ště} si napos\Ch{D}{led}
dám \Ch{G}{vodku} na zdra\Ch{E7}{ví,}

z očí \Ch{Ami}{pásku} {strhnu} \Ch{D7}{si,}
to abych \Ch{G}{viděl} na ne\Ch{Emi}{be}

a \Ch{Ami}{pak} vzpomenu \Ch{H7}{si,}
\Ch{Emi}{lásko}, na te\Ch{E7}{be,}
\Ch{Ami}{} \Ch{D}{} \Ch{G}{} \Ch{Emi}{}

a \Ch{Ami}{pak} vzpomenu \Ch{H7}{si na} te\Ch{Emi}{be.}
\ks

\zs
Až zítra ráno v pět přijde ke mně kněz,

řeknu mu, že se splet', že mně se nechce do nebes,

že žil jsem, jak jsem žil, a stejně tak i dožiju

a co jsem si nadrobil, to si i vypiju,

a co jsem si nadrobil, si i vypiju.
\ks

\zs
Až zítra ráno v pět poručík řekne: \uv{Pal!},

škoda bude těch let, kdy jsem tě nelíbal,

ještě slunci zamávám, a potom líto přijde mi,

že tě, lásko, nechávám, samotnou tady na zemi,

že tě, lásko, nechávám, na zemi.
\ks

\zs
Až zítra ráno v pět prádlo půjdeš prát

a seno obracet, já u zdi budu stát,

tak přilož na oheň a smutek v sobě skryj,

prosím, nezapomeň, nezapomeň a žij,

Lásko, nezapomeň a žij...
\ks

\kp























