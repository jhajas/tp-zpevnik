% -*-coding: utf-8 -*-

\zp{Píseň neznámého vojína}{Karel Kryl}

Rec: Zpráva z tisku: \uv{Obě delegace položily pak věnce na hrob Neznámého vojína.} A co na to Neznámý vojín?

\zs
V \Ch{Ami}{čele} klaka, pak ctnostné rodiny a náruč chryzan\Ch{E7}{tém},

černá saka a žena hrdiny pod paží s aman\Ch{Ami}{tem},\Ch{E7}{}

\Ch{Ami}{kytky} v dlaních a pásky smuteční civí tu před \Ch{E7}{branou},

ulpěl na nich pach síně taneční s bolestí sehra\Ch{Ami}{nou}.
\ks

\zr
Co tady \Ch{F}{čumíte}? Vlezte mi \Ch{G}{někam}!

Copak si \Ch{Ami}{myslíte}, že na to \Ch{G}{čekám}?

Co tady \Ch{Ami}{civíte}? Táhněte \Ch{G7}{domů}!

\Ch{Ami}{Pomníky} stavíte, \Ch{F}{prosím} vás, \Ch{E7}{komu}? \Ch{Ami}{} \Ch{E}{}
\kr

\zs
Jednou za čas se páni ustrnou a přijdou poklečet,

je to trapas, když s pózou mistrnou zkoušejí zabrečet,

pak se zvednou a hraje muzika písničku mizernou,

ještě jednou se trapně polyká nad hrobem s lucernou.
\ks

\zr
Co tady čumíte? Zkoušíte vzdechnout,

copak si myslíte, že jsem chtěl zdechnout?

Z lampasů je nám zle, proč nám sem leze?

Kašlu vám na fangle! Já jsem chtěl kněze!
\kr

\zs
Nejlíp je mi, když kočky na hrobě v noci se mrouskají,

ježto s těmi, co střílej' po sobě vůbec nic nemají,

mňoukaj' tence a nikdy neprosí, neslouží hrdinům,

žádné věnce pak na hrob nenosí Neznámým vojínům.

Kolik vám platějí za tenhle nápad?

Táhněte raději s děvkama chrápat!
\ks

\zr
Co mi to \Ch{Ami}{říkáte}? Že šel bych \Ch{G}{zas}? \Ch{G7}{Rád}?

\Ch{Ami}{Odpověď} čekáte? \Ch{F}{Nasrat}, jo, \Ch{E}{nasrat}! \Ch{Ami}{} \Ch{Ami6}{}
\kr

\kp





