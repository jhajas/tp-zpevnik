% -*-coding: utf-8 -*-

\zp{Cukrářská bossanova}{Jarek Nohavica}

\zs
Můj \Ch{Cmaj7}{přítel} \Ch{C#dim}{snídá} sedm \Ch{Dmi7}{kremrolí,} \Ch{G7}{}
a když je \Ch{Cmaj7}{spořádá}, dá si \Ch{C#dim}{repete},
\Ch{Dmi7}{cukrlát}\Ch{G7}{ko},

on totiž \Ch{Cmaj7}{říká:} \Ch{C#dim}{Dobré} lidi zuby \Ch{Dmi7}{nebolí,} \Ch{G7}{}
a je to \Ch{Cmaj7}{paráda}, chodit \Ch{C#dim}{po světě,}
a \Ch{Dmi7}{mít,} \Ch{G7}{mít} v ústech \Ch{Cmaj7}{sladko}. \Ch{C#dim}{} \Ch{Dmi7}{} \Ch{G7}{}
\ks


\zr
\Ch{Cmaj7}{Sláva,} \Ch{C#dim}{cukr} a \Ch{Dmi7}{káva} a půl litru \Ch{G7}{becherovky},

\Ch{Cmaj7}{hurá,} hurá, \Ch{C#dim}{hurá,} půjč mi \Ch{Dmi7}{bůra}, útrata dnes \Ch{G7}{dělá} čtyři stovky,

všechny \Ch{Cmaj7}{cukrářky} z celé \Ch{C#dim}{republiky}

na něho \Ch{Dmi7}{dělají} slaďounké \Ch{G7}{cukrbliky}

a on jim \Ch{Emi7}{za odmě}nu zpívá \Ch{A7}{zas} a znovu

\Ch{Dmi7}{tuhletu} \Ch{G7}{cukrářskou} bossa-no\Ch{Cmaj7}{vu.} \Ch{C#dim}{} \Ch{Dmi7}{} \Ch{G7}{}
\kr

\zs
Můj přítel Karel pije šťávu z bezinek,
říká, že nad ni není,
že je famózní, glukózní, monstrózní, ať si taky dám,

koukej, jak mu roste oblost budoucích maminek,
a já mám podezření,
že se zakulatí jako míč
a až ho někdo kopne, odkutálí se mi pryč,
a já zůstanu sám, úplně sám.
\ks

\zr  \kr

\zs
Můj přítel Karel Plíhal už na špičky si nevidí,
postava fortelná se mu zvětšuje,
výměra tři ary,

on ale říká: glycidy jsou pro lidi,
je prý v něm kotelna, ta cukry spaluje,
někdo se zkáruje, někdo se zfetuje
a on jí bonpari, bon, bon, bon, bonpari.
\ks

\zr  \kr

\kp





