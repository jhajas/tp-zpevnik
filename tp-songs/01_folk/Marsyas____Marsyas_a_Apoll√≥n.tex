% -*-coding: utf-8 -*-

\zp{Marsyas a Apollón}{Marsyas}

\zs
\Ch{D}{Ta} krásná dívka, \Ch{Emi}{co se} bojí o \Ch{A}{svoji} \Ch{Asus4}{krásu}

\Ch{D}{Athéna} \Ch{Emi7}{jméno} má \Ch{G}{za} starých \Ch{D}{dávných} časů

\Ch{D}{Odhodí} flétnu, \Ch{Emi}{hrát} nejde s \Ch{A}{naivnou} \Ch{Asus4}{tváří}

\Ch{D}{Ten}, kdo ji \Ch{Emi7}{najde} dřív, {tomu} se \Ch{D}{přání} zmaří
\ks

\zr
\Ch{Hmi}{Taky} Marsyas \Ch{A}{mámen} flétnou \Ch{Hmi}{věří}, že \Ch{G}{musí} přetnout

/: \Ch{D}{jedno} pravidlo, \Ch{Emi}{sázku} a \Ch{G}{hrát} \Ch{D}{líp} \Ch{A}{než} \Ch{D}{bůh} :/
\kr

\zs
Bláznivý nápad, snad nejvýš Marsyas míří

Apollón souhlasí, oba se s trestem smíří

Král Midas má říct, kdo je lepší, Apollón zpívá

O život soupeří, jen jeden vítěz bývá
\ks

\zr
Tak si Marsyas mámen flétnou věří a musí přetnout

/: jedno pravidlo, sázku a hrát líp než bůh :/
\kr

\zs
Obrátí nástroj, už ví, že nebude chválen

Prohrál a zápolí podveden vůlí krále

Sám v tichém hloučku, sám na strom připraví ráhno

Satyra k hrůze všech zaživa z kůže stáhnou
\ks

\zr
Tak si Apollón změřil síly, každý se musel mýlit

/: Nikdo nemůže kouzlit a hrát líp než bůh :/
\kr

\zs
Ta krásná dívka, co se bála o svoji krásu

Dárkyně moudrosti za starých dávných časů

Teď v tichém hloučku, v jejích rukou úroda, spása

Athéna jméno má, chybí jí tvář a krása
\ks

\zr
Dy dy dy .....
\kr

\kp























