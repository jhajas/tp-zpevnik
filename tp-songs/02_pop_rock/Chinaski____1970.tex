% -*-coding: utf-8 -*-

\zp{1970}{Chinaski}

/: \Ch{C}{} \Ch{Dmi}{} \Ch{F}{} \Ch{C}{} :/

\zs

\Ch{C}{Nevim} jestli je to \Ch{Dmi}{znát,}
možná by bylo lepší \Ch{Emi}{lhát,}

jsem silnej ročník \Ch{F}{sedmdesát,} tak začni počí\Ch{C}{tat.}

Nechci tu hloupě vzpomí\Ch{Dmi}{nat, koho} taky dneska zají\Ch{Emi}{má}

silnej ročník \Ch{F}{sedmdesát,} tak začni počí\Ch{C}{tat.}

Tenkrát tu bejval jinej \Ch{Dmi}{stát} a já byl blbej na kva\Ch{Emi}{drát,}

jsem silnej ročník \Ch{F}{sedmdesát,} třeba napří\Ch{G}{klad...}
\ks

\zr
\Ch{G}{Naši} mi vždycky říka\Ch{Ami}{li, jen nehas} co tě nepá\Ch{F}{lí,}

jakej pán takovej \Ch{C}{krám.}

\Ch{G}{Naši} mi vždycky říka\Ch{Ami}{li, co} můžeš sleduj z povzdá\Ch{F}{lí}

a nikdy nebojuj \Ch{C}{sám.}
\kr

\zs
Nevim jestli je to znám, možná by bylo lepší lhát,

jsem silnej ročník sedmdesát, nemohl jsem si vybírat.

Tak mi to přestaň vyčítat, naříkat co jsem za případ,

jsem silnej ročník sedmdesát a možná, že jsem rád.
\ks

\zr

\kr


\zs
/: Čas \Ch{F}{pádí,} čas letí, těžko ta \Ch{G}{léta} vrátíš zpět

a tak i \Ch{C}{Husákovy} děti dospěly \Ch{Dmi}{do Kristových} let. :/
\ks

\kp






